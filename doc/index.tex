\documentclass{article}  
\usepackage[utf8]{inputenc}

\title{Arquitecture Instantiation of ...}
\author{nilfurquimds}
\date{August 2016}

\usepackage{natbib}
\usepackage{graphicx}
\usepackage{enumitem}
\usepackage{tabularx}

\begin{document}

\maketitle

\section*{Introduction}
%REESCREVER
This document describes the requirements and architecture instantiation of RefSORS (Reference Architecture for Service Oriented Robotic Systems). which is a reference architecture based on Service-Oriented Architecture (SOA) specifically for indoor, grounded mobile robotic system. The instantiated architecture will be designed following the ArchSORS process. The resulting system will be a System-of-Systems composed by multiple robots transporting products on predetermined grid with guide lines dynamically recalcalculating the best path to navigate the industrial shop floor.

\subsection*{Objective}
The objective of the robotic system is to transport products on an industrial shop floor from a determined area to another. There will be multiple robots on the industrial shop floor, thus they will have to communictae to navigate effectively without collisions.
% \begin{figure}[h!]
% \centering
% \includegraphics[scale=.4]{./RA-2/broadSystemFlow.png}
% \caption{Broad System Flow}
% \label{fig:broadsystemflow}
% \end{figure}
% TODO: Add a service for controling the robot manually, like "Operation".\subsubsection{Camera Sensor} %CPI-IMAGE: Done
\section{Step RA-1: Application Characterization}

\subsection{Ambient Characterization}
The industrial shop floor is a plane area with guide lines painted black forming a grid. The grid assists the robots navigation. There are designed areas to pick up and deliver products, denominated product areas. Multiple product areas are positioned on two opposite sides of the grid. Each product area is given a unique identification. The product is designed to be picked up using a fork lift.

\subsection{Robot Characterization}
The robots are pioneers model P3-DX. The Pioneer P3-DX have two individually motorized wheels and a swivel for support, six sonars facing back and six sonars facing foward. Additionally a fork lift and a camera was equiped. The wheels enable for differential drive navigation used together with the camera to follow the guide lines of the industrial shop floor. The sonars will prevent collisions and assist product pick up. The product pick up is done with the fork lift.

\subsection{Functional Requirements}
The transport request informs all the robots the id of the product areas for pick up and for delivery. The robots must communicate with each other to select the closest robot to the product area determined for product pick up, based on Manhattan distance.%"""(consider or not position of actual robots on the grid)"""
The robots must be able to follow the guide lines on the floor. At all times navigating the industrial shop floor the robots will have to communicate with each other to dynamically calculate best path, always avoid full stop and never colliding, given that other robots will be moving on the grid and two robots cannot past by each other while following the same guide line.

\newlist{FR}{enumerate}{1}
\setlist[FR]{label=FR\arabic*:}
\begin{FR}
	
	



	%SYSTEM
	\item The system must be able to take requests. %RR1 %Application
	\item The system must be able to transport products from one PA to another. %RR1 %Application
	\item The system must work decentralized. %RR3 %Application
	\item The system must be able to choose the closest Manhattan distance robot to the product for the job. %RR3, RR5 %Application
	\item The system must avoid full stop of robots. %RR15 %Navigation 
	\item The system must transport products simultaneously if there are robots avaible. %RR1, RR3, RR5 %Application 
	\item The robots must operate individually. No robot must stop if any robot stops. They must still be able to deliver product and take requests. *Exception if robots block other robots' path and there are no other avaible paths. %RR3, RR5, RR8 %Application %ROBOTS
	\item Robots must be able to control differential drive. %RR29 %Drive Actuator
	\item Robots must be able control the fork lift. %RR29 %Fork Lift Actuator
	\item Robots must be able process images captured by the camera. %RR19, RR24 %Camera Sensor
	\item Robots must be able process 16 sonar data from the front and back of the robot. %RR24 %Sonar Sensor
	\item Robots must know the grid. %RR14, RR16, RR17 %Mapping
	\item Robots must have a graph representing the grid. %RR16, RR17, RR20 %Mapping
	\item Robots must identify the guide lines using images captured by the camera. %RR19 %Image Processing
	\item Robots must be able to follow the guide lines of the grid. %RR15 %Navigation
	\item Robots must be able know where are the product areas. %RR14 %Mapping
	\item Robots must be able know their initial position. %RR14 %Localization
	\item Robots must be able know their aproximate positon based on the grid and as they go by crossings. %RR14 %Localization
	\item Robots must be able to lift the product. %RR13 %Object Manipulation
	\item Robots, once close by, must use sonars to locate the product. %RR13 %Object Manipulation
	\item Robots must avoid collision with other robots. %RR15 %Navigation
	\item Robots must be able to communicate with each other. %RR11, RR21(RETIRADO)%Interaction
	\item Robots must be able know the position of other robots. %RR11, RR21(RETIRADO), RR16 %Mapping
	\item Robots must be able to calculate shortest path using the dijkstra algorithm. %RR16 %Path Planning
	\item Robots must be able calculate the shortest path considering the posistion of other robots. %RR5, RR11 %Path Planning %SEM DOMAIN CONCEPT
	\item Robots must be able to notify when done transporting. %RR12 %Control
	\item Robots must release products in vertical position. %RR13 %Object Manipulation
	\item Robots must be able to be controled be an operator if needed. %RR8, RR12 %Robot Agent?? %TRANSPORT ORDER
	\item The transport order must be done by an operator on a remote computer. %RR12 %Application
	\item The transport order must inform the PAs where the product must be collected and where it must be delivered. %RR12 %Application
	\item The transport order must be broadcast to all robots. %RR3 %Application
\end{FR}
\section{Step RA-2: Capability Identification}
In this step the flows of different views of the system are modeled using BPMN diagrams. Then capabilities are identified by mapping these functional requirements with RefSORS requirements and analyzing the each flow. Finally these capabilities are assessed based on their functionalities to decide which are going to be exposed as services and which are going to be provided as components that support these services.

\subsection{System Flow Model}
From functional requirements and concepts from the reference architecture BPMN diagrams were modeled to represent the application flow. The modeled application flow was divided in six BPMN diagrams and seven polls shown in order of occurrence, each lane represents a possible capability and tasks are inspired by the robotic system required functionalities.
Figure 1 shows the most abstract flow, the flow of the broadest system. Each robot initializes and requests the map to the master, that delivers it. Then they determine their own positions. After they know their own positions, the master can process the product transport request. It asks the distance of the robots from the product, chooses the closest one, and continues to wait for other requests.

%Broad System Flow
\begin{figure}[ht!]
 \centering
 \includegraphics[scale=.23]{./RA-2/broadSystemFlow.png}
 \caption{Figure 1: Broad System Flow}
 \label{fig:broadsystemflow}
\end{figure}

Figure 1: This BPMN diagram shows how the individual robots decide wether to answer to a transport request or not.

The robot receives a transport request. After the request is received, it gets the product's position and it's own position in the map then calculates the Manhattan distance between them and broadcasts it. The distance calculated by it is compared to the distances calculated by other robots, if the former is smaller it broadcasts a message saying that it will take the request. If there is any robot closer to the product it waits for another request.



%Move to Destination Flow
\begin{figure}[ht!]
 \centering
 \includegraphics[scale=.282]{./RA-2/moveToDestination.png}
 \caption{Figure 2: Move to Destination Robot Flow}
 \label{fig:movetodestination}
\end{figure}
Figure 2: This BPMN diagram shows how the robot moves to a destination.

Upon a need to navigate to a destination on the map, the robot needs it's own position and the other robots positions. When this positions needed are aquired, it calculates the best path using a dijkstra algorithm but removing the pathways blocked by other robots. Then while a path recalculation triggering event is waited the last calculated path is executed. The execution of the path is done by following the guide line until a crossing is found, then it chooses wether to go straigth, turn left or turn right based on the calculated path. Everytime the robot goes by a crossing it broadcasts it's location.

%Pick Up Product Flow
\begin{figure}[ht!]
 \centering
 \includegraphics[scale=.276]{./RA-2/pickup.png}
 \caption{Figure 3: Pick Up Product Flow}
 \label{fig:pickup}
\end{figure}

Figure 3: This BPMN diagram shows how the robot picks up a product.

When the robot reaches the product's position, meaning the product is close by, the fork is moved to predetermined height and the sonars are used to find where the product is. Then the robot moves to position the fork under the product. It repeates the sensing and moving process until the fork is in a satisfying place when it lifts the product.

%Release Product
\begin{figure}[ht!]
 \centering
 \includegraphics[scale=.33]{./RA-2/release.png}
 \caption{Figure 4: Release Product Flow}
 \label{fig:release}
\end{figure}

Figure 4: This BPMN diagram shows the process of releasing the product on the ground.

Once the product reaches destination the robot stops and starts the product drop off. The fork is lower until product reaches the ground and the robot moves straight backward until the fork is far away enough from the product so it can move it to navigation position.

%Subscribe to Sensors Flow
\begin{figure}[ht!]
 \centering
 \includegraphics[scale=.45]{./RA-2/subscribeSensors.png}
 \caption{Figure 4: Subscribe to Sensors Flow}
 \label{fig:subscribeSensors}
\end{figure}

Figure 5: This BPMN diagram shows the generic flow of subscribing to sensors and handling their data.

This flow happens during the robot's initialization. After subscribing to sonars topic the distances are get and handled, simultaneously the distances are updated. The same thing happens to the camera, but instead of distances images are handled.

%Follow Guide Line Flow
\begin{figure}[ht!]
 \centering
 \includegraphics[scale=.359]{./RA-2/followLine.png}
 \caption{Figure 1: Follow Guide Line Flow}
 \label{fig:followline}
\end{figure}

Figure 6: This BPMN diagram shows how the robot follows the guide line.

To follow the guide line the robot uses the camera, it gets the image and processes it. If the line is facing right or is translated to the right it turns right, if the line is facing forward and is at the center it goes straingth else if the line is facing left or translated to the left it turns left. If the line is facing one orientation and translated to the other it takes the decision based on the translation.

\subsection{Decompose Robot Application}

The BPMN diagrams are used to indentify capabilities. The process of analysing the BPMN diagrams to identify capabilities entails in comparing the BPMN to the functional requirements and RefSORS requirements. Due to how the system was modeled using the diagrams, most BPMN lanes are potential capabilities. Then the potential capabilities are compared to functional requirements and RefSORS requirements, if they fulfill at least part of a functional or RefSORS requirement they go to the next step. They next step is compare them to the domain concepts and RefSORS capabilities, if they are related they become capabilities.

%Robotic Application: 1..9 | 1 2 3 4 5 6 7 8 9
%Robotic Agent: 6..9 | 6 7 8 9
%Task: 10..19 | 10 11 12 13 14 15 16 17 18 19
%Knowledge: 20..23 | 20 21 22 23
%Device Driver: 24..30 | 24 25 26 27 28 29 30
%General: 31 32
{
\centering
\begin{tabular}{| r | p{2cm} | c | c |}
	\hline
	Functional Requirement & RefSORS Requirement & Domain Concept & Capability \\ 
	\hline
	FR1, FR2, FR6 & R-R1 	& Application & Application \\
							%& R-R2 	& Application & \\
	FR3, FR4, FR7, FR31 	& R-R3 & Application & Application\\
	FR23 					& R-R3 & Application & Mapping\\ %??
	FR22 					& R-R3 & Application & Control\\ %??
							%& R-R4 	& Application & \\
	FR6, FR7 				& R-R5 & Application & Application\\
							%& R-R6 	& Robotic Agent & \\
							%& R-R7 	& Robotic Agent & \\
	\hline
	FR28 					& R-R8 & Robotic Agent & Robot Agent\\
							%& R-R9 	& Robotic Agent & \\
							%& R-R10 & Task & \\
	\hline
	FR12 					& R-R11 & Task & Mapping\\
	FR22 					& R-R11 & Task & Control\\
	FR26				 	& R-R12 & Task & Control\\
	FR28				 	& R-R12 & Task & Robot Agent\\%Robot agent?
	FR29, FR30			 	& R-R12 & Task & Application\\%Application? (Operation no edusig)
	FR19, FR20, FR27		& R-R13 & Task & Object Manipulation\\
	FR12, FR16				& R-R14 & Task & Mapping\\
	FR17, FR18 				& R-R14 & Task & Localization\\
	FR12 					& R-R15 & Task & Mapping\\
	R15, FR21 				& R-R15 & Task & Navigation\\
	FR23, FR35				& R-R15 & Task & Path Planning\\
	FR12, FR13 				& R-R16 & Task & Mapping\\
	FR24, FR23				& R-R16 & Task & Path Planning\\
	FR12, FR13 				& R-R17 & Task & Mapping\\
							%& R-R18 & Task & \\
	FR10 					& R-R19 & Task & Camera Sensor\\
	FR14	 				& R-R19 & Task & Image Processing\\
	\hline
	FR11 					& R-R20 & Knowledge & Sonar Sensor\\
	FR13 					& R-R20 & Knowledge & Mapping\\ %Map information?
	FR22 					& R-R21 & Knowledge & Control\\
	FR23 					& R-R21 & Knowledge & Mapping\\
							%& R-R22 & Knowledge & \\
							%& R-R23 & Knowledge & \\
	\hline
	FR10 					& R-R24 & Device Driver & Camera Sensor\\
	FR11 					& R-R24 & Device Driver & Sonar Sensor\\
							%& R-R25 & Device Driver & \\
	FR22 					& R-R26 & Device Driver & Control\\
							%& R-R27 & Device Driver & \\
							%& R-R28 & Device Driver & \\
	FR8 					& R-R29 & Device Driver & Drive Actuator\\
	FR9 					& R-R29 & Device Driver & For Lift Actuator\\
	\hline
\end{tabular}
}


%Capabilities
\begin{figure}[ht!]
 \centering
 \includegraphics[scale=.3]{./RA-2/capabilityDiagram.png}
 \caption{Figure 9: Capabilites}
 \label{fig:capabilities}
\end{figure}

Figure 9: The diagram shows the identified capabilities, with their functionalities and relationships with each other.

\subsection{Rationalize Capabilities}
After identifying the capabilities, they must be analyzed to decide whether they will become services. Capabilities might also join other capabilities to form a service if they are too simple. They might also form more than one service if they are too complex. Table 2.3 shows the analysis of each capability and if they became one or more services or not at all.

{
\centering
\begin{tabular}{| l | p{6.6cm} | c |}
	\hline
	Capability & Analysis & Service\\
	\hline
	Camera Sensor & Needed to communicate with camera sensor. & yes\\
	Sonar Sensor & Needed to communicate with sonar sensor. & yes\\
	Fork Lift Actuator & Needed to communicate with fork lift actuator. & yes\\
	Drive Actuator & Needed to communicate with drive actuator. & yes\\
	\hline
	Image Processing & Needed to process images to identify the guide lines. & yes\\
	Mapping & Grid mapping to navigate and know other robots' positions. &\\
	Localization & Know it's own position on the grid. &\\ %(maybe too simple, join with mapping?, analyze scalability. Possible to create more complex localization?)
	Navigation & Navigate the grid, although simple, needs to be concise in case of change of logic and maintenance. & yes\\
	Path Planning & Calculate path to destination. & yes\\
	Object Manipulation & Logic of picking up and releasing product. & yes\\
	Control & Control unit. & yes\\
	Robotic Agent & Abstration of the robots functionalities. & yes\\
	Application & Interact with operator, send transport messages. & yes\\
	\hline
\end{tabular}
}


\section{Step RA-3: Robotic Architecture Modeling}
\subsection{Robotic Services Specification}
%Descição do que o serviço faz
\subsection{Robotic Services Modeling}
%Contrato, interface e protocolo
\subsection{Robotic Service Composition}

\section{Step RA-4: Robotic Architecture Detailing}
\subsection{Implementation Strategies}
\subsubsection*{Camera Sensor}
\subsubsection*{Sonar Sensor}
\subsubsection*{Fork Lift Actuator}
\subsubsection*{Drive Actuator}
\subsubsection*{Image Processing}
\subsubsection*{Mapping}
\subsubsection*{Localization}
\subsubsection*{Navigation}
\subsubsection*{Path Planning}
\subsubsection*{Object Manipulation}
\subsubsection*{Communication}
\subsubsection*{Operation}
\subsubsection*{Control}
\subsubsection*{Robotic Agent}
\subsubsection*{Application}

\subsection{Deployment}

\bibliographystyle{plain}
\bibliography{references}
\end{document}
