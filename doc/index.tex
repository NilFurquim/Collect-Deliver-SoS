\documentclass{article}  
\usepackage[utf8]{inputenc}

\title{Architecture Instantiation of ...}
\author{nilfurquimds}
\date{August 2016}

\usepackage{natbib}
\usepackage{graphicx}
\usepackage{enumitem}

\begin{document}

\maketitle

\section*{Introduction}
%REESCREVER
This document describes the requirements and architecture instantiation of RefSORS (Reference Architecture for Service Oriented Robotic Systems). which is a reference architecture based on Service-Oriented Architecture (SOA) specifically for indoor, grounded mobile robotic system. The instantiated architecture will be designed following the ArchSORS process. The resulting system will be a System-of-Systems composed by multiple robots transporting products on predetermined grid with guide lines dynamically recalculating the best path to navigate the industrial shop floor.

\subsection*{Objective}
The objective of the robotic system is to transport products on an industrial shop floor from a determined area to another. There will be multiple robots on the industrial shop floor, thus they will have to communicate to navigate effectively without collisions.
% \begin{figure}[h!]
% \centering
% \includegraphics[scale=.4]{./RA-2/broadSystemFlow.png}
% \caption{Broad System Flow}
% \label{fig:broadsystemflow}
% \end{figure}
% TODO: Add a service for controlling the robot manually, like "Operation".\subsubsection{Camera Sensor} %CPI-IMAGE: Done
\section{Step RA-1: Application Characterization}

\subsection{Ambient Characterization}
The industrial shop floor is a plane area with guide lines painted black forming a grid. The grid assists the robots navigation. There are designed areas to pick up and deliver products, denominated product areas. Multiple product areas are positioned on two opposite sides of the grid. Each product area is given a unique identification. The product is designed to be picked up using a fork lift.

\subsection{Robot Characterization}
The robots are pioneers model P3-DX. The Pioneer P3-DX have two individually motorized wheels and a swivel for support, six sonars facing back and six sonars facing foward. Additionally a fork lift and a camera was equiped. The wheels enable for differential drive navigation used together with the camera to follow the guide lines of the industrial shop floor. The sonars will prevent collisions and assist product pick up. The product pick up is done with the fork lift.

\subsection{Functional Requirements}
The transport request informs all the robots the id of the product areas for pick up and for delivery. The robots must communicate with each other to select the closest robot to the product area determined for product pick up, based on Manhattan distance.%"""(consider or not position of actual robots on the grid)"""
The robots must be able to follow the guide lines on the floor. At all times navigating the industrial shop floor the robots will have to communicate with each other to dynamically calculate best path, always avoid full stop and never colliding, given that other robots will be moving on the grid and two robots cannot past by each other while following the same guide line.

\newlist{FR}{enumerate}{1}
\setlist[FR]{label=FR\arabic*:}
\begin{FR}
	
	



	%SYSTEM
	\item The system must be able to take requests. %RR1 %Application
	\item The system must be able to transport products from one PA to another. %RR1 %Application
	\item The system must work decentralized. %RR3 %Application
	\item The system must be able to choose the closest Manhattan distance robot to the product for the job. %RR3, RR5 %Application
	\item The system must avoid full stop of robots. %RR15 %Navigation 
	\item The system must transport products simultaneously if there are robots avaible. %RR1, RR3, RR5 %Application 
	\item The robots must operate individually. No robot must stop if any robot stops. They must still be able to deliver product and take requests. *Exception if robots block other robots' path and there are no other avaible paths. %RR3, RR5, RR8 %Application %ROBOTS
	\item Robots must be able to control differential drive. %RR29 %Drive Actuator
	\item Robots must be able control the fork lift. %RR29 %Fork Lift Actuator
	\item Robots must be able process images captured by the camera. %RR19, RR24 %Camera Sensor
	\item Robots must be able process 16 sonar data from the front and back of the robot. %RR24 %Sonar Sensor
	\item Robots must know the grid. %RR14, RR16, RR17 %Mapping
	\item Robots must have a graph representing the grid. %RR16, RR17, RR20 %Mapping
	\item Robots must identify the guide lines using images captured by the camera. %RR19 %Image Processing
	\item Robots must be able to follow the guide lines of the grid. %RR15 %Navigation
	\item Robots must be able know where are the product areas. %RR14 %Mapping
	\item Robots must be able know their initial position. %RR14 %Localization
	\item Robots must be able know their aproximate positon based on the grid and as they go by crossings. %RR14 %Localization
	\item Robots must be able to lift the product. %RR13 %Object Manipulation
	\item Robots, once close by, must use sonars to locate the product. %RR13 %Object Manipulation
	\item Robots must avoid collision with other robots. %RR15 %Navigation
	\item Robots must be able to communicate with each other. %RR11, RR21(RETIRADO)%Interaction
	\item Robots must be able know the position of other robots. %RR11, RR21(RETIRADO), RR16 %Mapping
	\item Robots must be able to calculate shortest path using the dijkstra algorithm. %RR16 %Path Planning
	\item Robots must be able calculate the shortest path considering the posistion of other robots. %RR5, RR11 %Path Planning %SEM DOMAIN CONCEPT
	\item Robots must be able to notify when done transporting. %RR12 %Control
	\item Robots must release products in vertical position. %RR13 %Object Manipulation
	\item Robots must be able to be controled be an operator if needed. %RR8, RR12 %Robot Agent?? %TRANSPORT ORDER
	\item The transport order must be done by an operator on a remote computer. %RR12 %Application
	\item The transport order must inform the PAs where the product must be collected and where it must be delivered. %RR12 %Application
	\item The transport order must be broadcast to all robots. %RR3 %Application
\end{FR}
\section{Step RA-2: Capability Identification}
In this step the flows of different views of the system are modeled using BPMN diagrams. Then capabilities are identified by mapping these functional requirements with RefSORS requirements and analyzing each flow. Finally these capabilities are assessed based on their functionalities to decide which are going to be exposed as services and which are going to be provided as components that support these services.

\subsection{System Flow Model}
From functional requirements and concepts from the reference architecture BPMN diagrams were modeled to represent the application flow. The modeled application flow was divided in six BPMN diagrams and seven pools shown in order of occurrence, each lane represents a possible capability and tasks are inspired by the robotic system required functionalities.
Figure 1 shows the most abstract flow, the flow of the broadest system. First of all every robot must subscribe to their sensors. Second they wait for a product request. The product request requires the robots to decide which is closest to the product to be delivered. The closest robot then transports the product and waits for another request.

%Broad System Flow
\begin{figure}[ht!]
 \centering
 \includegraphics[scale=.23]{./RA-2/broadSystemFlow.png}
 \caption{Broad System Flow}
 \label{fig:broadsystemflow}
\end{figure}

Figure \ref{fig:broadsystemflow}: This BPMN diagram shows how the individual robots decide whether to answer to a transport request or not.

The robot receives a transport request. After the request is received, it gets the product's position and it's own position in the map then calculates the Manhattan distance between them and broadcasts it. The distance calculated by it is compared to the distances calculated by other robots, if the former is smaller it broadcasts a message saying that it will take the request. If there is any robot closer to the product it waits for another request.

%Move to Destination Flow
\begin{figure}[ht!]
 \centering
 \includegraphics[scale=.282]{./RA-2/moveToDestination.png}
 \caption{Move to Destination Robot Flow}
 \label{fig:movetodestination}
\end{figure}
Figure \ref{fig:movetodestination}: This BPMN diagram shows how the robot moves to a destination.

Upon a need to navigate to a destination on the map, the robot needs it's own position and the other robots positions. Once the positions needed are acquired, it calculates the best path using a dijkstra algorithm but removing the pathways blocked by other robots. Then while a path recalculation triggering event is waited the last calculated path is executed. The execution of the path is done by following the guide line until a crossing is found, then it chooses whether to go straight, turn left or turn right based on the calculated path. Every time the robot goes by a crossing it broadcasts it's new location on the grid.

%Pick Up Product Flow
\begin{figure}[ht!]
 \centering
 \includegraphics[scale=.276]{./RA-2/pickup.png}
 \caption{Pick Up Product Flow}
 \label{fig:pickup}
\end{figure}

Figure \ref{fig:pickup}: This BPMN diagram shows how the robot picks up a product.

When the robot reaches the product's position, meaning the product is close by, the fork is moved to predetermined height and the sonars are used to find where the product is more specifically. Then the robot moves to position the fork under the product. It repeats the sensing and moving process until the fork is in a satisfying place when it lifts the product.

%Release Product
\begin{figure}[ht!]
 \centering
 \includegraphics[scale=.33]{./RA-2/release.png}
 \caption{Release Product Flow}
 \label{fig:release}
\end{figure}

Figure \ref{fig:release}: This BPMN diagram shows the process of releasing the product on the ground.

Once the product reaches destination the robot stops and starts the product drop off. The fork is lowered until the product reaches the ground and the robot moves straight backward until the fork is far away enough from the product so it can lift the fork back up.

%Subscribe to Sensors Flow
\begin{figure}[ht!]
 \centering
 \includegraphics[scale=.45]{./RA-2/subscribeSensors.png}
 \caption{Subscribe to Sensors Flow}
 \label{fig:subscribeSensors}
\end{figure}

Figure \ref{fig:subscribeSensors}: This BPMN diagram shows the generic flow of subscribing to sensors and handling their data.

This flow happens during the robot's initialization. After subscribing to sonars topic the distances are handled, simultaneously the distances are updated. The same thing happens to the camera, but instead of distances images are handled.

%Follow Guide Line Flow
\begin{figure}[ht!]
 \centering
 \includegraphics[scale=.359]{./RA-2/followLine.png}
 \caption{Follow Guide Line Flow}
 \label{fig:followline}
\end{figure}

Figure \ref{fig:followline}: This BPMN diagram shows how the robot follows the guide line.

To follow the guide line the robot uses the camera, it gets the image and processes it. If the line is facing right or is translated to the right it turns right, if the line is facing forward and is at the center it goes straight else if the line is facing left or translated to the left it turns left. If the line is facing one orientation and translated to the other it takes the decision based on the translation.

\subsection{Decompose Robot Application}

The BPMN diagrams are used to identify capabilities. The process of analyzing the BPMN diagrams to identify capabilities entails in comparing the BPMN to the functional requirements and RefSORS requirements. Due to how the system was modeled using the diagrams, all BPMN lanes are potential capabilities. Then the potential capabilities are compared to functional requirements and RefSORS requirements, if they fulfill at least part of a functional or RefSORS requirement they go to the next step. They next step is compare them to the domain concepts and RefSORS capabilities, if they are related they become capabilities.

%Robotic Application: 1..9 | 1 2 3 4 5 6 7 8 9
%Robotic Agent: 6..9 | 6 7 8 9
%Task: 10..19 | 10 11 12 13 14 15 16 17 18 19
%Knowledge: 20..23 | 20 21 22 23
%Device Driver: 24..30 | 24 25 26 27 28 29 30
%General: 31 32
\begin{table}
	\footnotesize
	\centering
	\begin{tabular}{| r | p{1.8cm} | c | c |}
		\hline
		Functional Requirement & RefSORS Requirement & Domain Concept & Capability \\ 
		\hline
		FR1, FR2, FR6 			& RR1 & Application & Application \\
		%& RR2 & Application & Robotic Agent or Application \\
		FR3, FR4, FR6, FR7, FR31& RR3 & &\\
		%& RR4 & Application & Application\\
		FR4, FR6, FR7, FR25 	& RR5 & &\\
		\hline
		%& RR6 & Robotic Agent \\
		%& RR7 & Robotic Agent or Control 	\\
		FR7 	& RR8 & Robotic Agent & Control \\%or Robotic Agent \\
		FR28 	& RR8 & & Robotic Agent \\%or Control \\
		%& RR9 & Robotic Agent or Control\\
		%& RR10 & Interaction \\
		\hline
		FR22, FR23, FR25 		& RR11 & Task & Interaction \\
		FR26, FR29, FR30 		& RR12 & 	  & \\
		FR28 			 		&      & 	  & Operation \\
		FR19, FR20, FR27 		& RR13 & 	  & Object Manipulation\\
		FR12, FR16, FR17 FR18 	& RR14 & 	  & Localization\\
		FR5, FR15, FR21 		& RR15 & 	  & Navigation\\
		FR12, FR13, FR23, FR24 	& RR16 & 	  & Path Planning\\
		% FR12, FR13 & RR17 & Task & Mapping \\
		%& RR18 & Mapping \\
		FR10, FR14 				& RR19 & 	  & Image Processing \\ %Image Processing = support or Path Planning or Mapping or Localization 	\\
		\hline
		FR13 & RR20 & Knowledge & Map Information \\
		%& RR21 & Robotic Agent or Application or Knowledge or Path Planning or Mapping or Localization 	 \\
		%& RR22 & Knowledge \\
		%& RR23 & Knowledge \\
		\hline
		FR10 & RR24 & Device Driver & Sonar Sensor\\
		FR11 & 		& 				& Camera Sensor\\
		%& RR25 & Sensor Driver\\
		%& RR26 & Sensor Driver\\
		%& RR27 & Resource Driver\\
		%& RR28 & Resource Driver\\
		FR8 & RR29 	& & Drive Actuator\\
		FR9 & 		& & Forklift Actuator\\
		%& RR30 & Actuator Driver\\
	 	%& RR31 &\\
	 	%& RR32 &\\
	 	\hline
	\end{tabular}
	\caption{Functional Requirements classification}
	\label{tab:fr_classification}
\end{table}


%Capabilities 
%TODO: Colocar funções diferentes na capability Sonar sensor. Funções mais "importantes", vide RA-4.
\begin{figure}[ht!]
 \centering
 \includegraphics[scale=.42]{./RA-2/capabilityDiagram.png}
 \caption{Capabilities}
 \label{fig:capabilities}
\end{figure}

Figure \ref{fig:capabilities}: The diagram shows the identified capabilities, with their functionalities and relationships with each other.

\subsection{Rationalize Capabilities}
After identifying the capabilities, they must be analyzed to decide whether they will become services. Capabilities might also join other capabilities to form a service if they are too simple. They might also form more than one service if they are too complex. Table \ref{tab:fr_classification} shows the analysis of each capability and if they became one or more services or not at all. 

\begin{table}[ht]
	\footnotesize
	\centering
	\begin{tabular}{| l | p{3.2cm} | p{3.2cm} | c |}
		\hline
		Capability & Too simple & Too complex & Service\\
		\hline
		Camera Sensor 		& No, deals with camera sensor specifics. & No, deals only with camera sensor. & yes\\
		\hline
		Sonar Sensor 		& No, deals with sonar sensor specifics. & No, deals only with sonar sensors. & yes\\
		\hline
		Fork Lift Actuator 	& No, deals with forklift actuator specifics. & No, deals only with forklift actuator. & yes\\
		\hline
		Drive Actuator 		& No, deals with drive actuator specifics. & No, deals with drive actuator specifics. & yes\\
		\hline
		Image Processing 	& No, processes images generated by the camera. & No, cohesive. & yes\\
		\hline
		MapInformation		& No, deals with map representation. & No, cohesive. & yes \\
		\hline
		Localization 		& No, defines the actual position of the robot analyzing as it goes by crossings. & No, cohesive. & yes\\ %(maybe too simple, join with Navigation?, analyze scalability. Possible to create more complex localization?)
		\hline
		Navigation 			& No, more abstract movement of the robot & No, cohesive. & yes\\
		\hline
		Path Planning 		& No, calculates shortest path using two different algorithms. & No, cohesive. & yes\\
		\hline
		Object Manipulation & No, abstracts picking and releasing of product. & No, cohesive. & yes\\
		\hline
		Communication 		& No, logic to communicate with other robots. & No, cohesive. & yes\\
		\hline
		Operation	 		& No, controls the robot manually. & No, cohesive. & yes\\
		\hline
		Control 			& No, talks to several other capabilities. & No, only deals with logic. & yes\\
		\hline
		Robotic Agent 		& No, different activities done by the robot. & No, only abstracts robots activities. & yes\\
		\hline
		Application 		& No, abstraction of the whole application of transporting products. & No, cohesive. & yes\\
		\hline
	\end{tabular}
	\caption{Capabilities rationalization table}
\end{table}

The interaction will be abstracted into the other capabilities in the project.
\section{Step RA-3: Robotic Architecture Modeling}
\subsection{Robotic Services Specification}
%Descição do que o serviço faz
\subsubsection*{Camera Sensor}
Instance of sensor driver service. This service is responsible for configuring the camera and capturing images. Images can be captured given an instant or periodicaly.
%TODO: camera model and specifics. Since it is simulated the camera is yet to be chosen.
\\Related Requirements:

\subsubsection*{Sonar Sensor}
Instance of sensor driver service. This service is responsible for configuring the sonar sensors and get the distances gathered by the sensors. Distances can be gathered given an instant or periodicaly.
\\Related Requirements:

\subsubsection*{Fork Lift Actuator}
Instance of actuator driver. This service is responsible for configuring and raising and lowering the fork lift.
\\Related Requirements:

\subsubsection*{Drive Actuator}
Instace of actuator driver. This service is responsible for configuring and controlling differential drive.
\\Related Requirements:

\subsubsection*{Image Processing}
Instance of support service. This service is responsible for gathering images from the camera sensor and process it to provide meaninful information, such as the guide lines positions and crossings.
\\Related Requirements:

\subsubsection*{MapInformation}
This service is responsible for keeping a map of the grid, represented as a graph. The map will also keep the localization of the other robots, that will be dynamicaly updated.
\\Related Requirements:

\subsubsection*{Localization}
This service is responsible for keeping the current location of the robot in the map. 
\\Related Requirements:

\subsubsection*{Navigation}
This service is responsible for navigating the map and following the guide lines of the grid. 
\\Related Requirements:

\subsubsection*{Path Planning}
This service is responsible for calculating paths using two algorithms, Manhatan Distance and Dijkstra. 
\\Related Requirements:

\subsubsection*{Object Manipulation}
Abstraction of object manipulation. This service is responsible for getting the robot to the right position and picking up and realeasing the product.
\\Related Requirements:

\subsubsection*{Communication}
This service is responsible for communicating with other robots, sending and receving messages. It will also turn raw data into meaninful information such as a transport request and position information of other robots. 
\\Related Requirements:

\subsubsection*{Control}
This service is responsible for taking logical decision of the robots in a broad view.
\\Related Requirements:

\subsubsection*{Robotic Agent}
Abstraction of the robots' functionalities. This service is responsible for providing abstraction to all robots functionalities, such as transporting a product.%TODO: Checkar capability diagram.
\\Related Requirements:

\subsubsection*{Application}
This service is the application itself. It provides a communication between the operator and the robots to do the task of transporting products.
\\Related Requirements:

\subsection{Robotic Services Modeling}
%Contrato, interface e protocolo
%TODO: Check all CPI to see if handler is dealing with all cases.
\subsubsection{Camera Sensor} %CIP-IMAGE: Done
Figure \ref{fig:camerasensor_cip} describes camera sensor service's contract, interface and protocol. The camera sensor service provides two funcionalities, getImage and subscribe. The functionality getImage provides the image captured by the camera in that instant. The functionality subscribe provides images periodically, which requires the implementation of a camerasensorhandler to handle the periodical image updates.
\begin{figure}[ht!]
 \centering
 \includegraphics[scale=.43]{./RA-3/cameraSensor_CIP.png}
 \caption{Camera Sensor}
 \label{fig:camerasensor_cip}
\end{figure}

\subsubsection{Sonar Sensor} %CIP-IMAGE: Done
Figure \ref{fig:sonarsensor_cip} describes sonar sensor service's contract, interface and protocol. The sonar sensor service provides two functionalities, getDistance and subscribe. The functionality getDistance gives the distances captured by each sonar unity in that instant. The functionality subscribe provides the distances of each sonar unity updated periodically, requires the implementation of sonarSensorHandler to deal with the periodical updates.
\begin{figure}[ht!]
 \centering
 \includegraphics[scale=.43]{./RA-3/sonarSensor_CIP.png}
 \caption{Sonar Sensor}
 \label{fig:sonarsensor_cip}
\end{figure}

\subsubsection{Fork Lift Actuator} %CIP-IMAGE: Done
Figure \ref{fig:forkliftactuator_cip} describes fork lift actuator service's contract, interface and protocol. The fork lift actuator service provides four functionalities, lower, raise, stop and raise. The functionalities lower and raise are responsible for moving the fork down and up, respectively, requiring the distance to be moved and returning the success of the action. The functionality stop stops any action occuring. The functionality report reports the actual position of the fork.
\begin{figure}[ht!]
 \centering
 \includegraphics[scale=.48]{./RA-3/forkLiftActuator_CIP.png}
 \caption{Fork Lift Actuator}
 \label{fig:forkliftactuator_cip}
\end{figure}

\subsubsection{Drive Actuator} %CIP-IMAGE: Done
Figure \ref{fig:driveactuator_cip} describes drive actuator service's contract, interface and protocol. The drive actuator service provides two functionalities, drive and report. The functionality drive moves the wheels according to the linear and angular velocity provided. The functionality report reports the position of the wheel at that instant.
\begin{figure}[ht!]
 \centering
 \includegraphics[scale=.48]{./RA-3/driveActuator_CIP.png}
 \caption{Drive Actuator}
 \label{fig:driveactuator_cip}
\end{figure}

\subsubsection{Image Processing} %CIP-IMAGE: Done* parameters for handleProcessedImage() missing
Figure \ref{fig:imageprocessing_cip} describes image processing service's contract, interface and protocol. The image processing service provides two funcionalities, identifyGuideLines and subscribe. The functionality identifyGuideLines takes an image and returns the guide lines identified at that instant. The funcionality subscribe provides processed images periodically, requires a imageProcessingHandler to deal with the periodical processed images updates. 
\begin{figure}[ht!]
 \centering
 \includegraphics[scale=.43]{./RA-3/imageProcessing_CIP.png}
 \caption{Image Processing}
 \label{fig:imageprocessing_cip}
\end{figure}

\subsubsection{MapInformation} %CIP-IMAGE: nope
% \begin{figure}[ht!]
%  \centering
%  \includegraphics[scale=.43]{./RA-3/mapInformation_CIP.png}
%  \caption{Mapping}
%  \label{fig:mapinformation_CIP}
% \end{figure}

\subsubsection{Localization} %CIP-IMAGE: Done* missing return for getPosition()
Figure \ref{fig:localization_cip} describes localization service's contract, interface and protocol. The localization service provides two functionalites, getPosition and subscribe. The functionality getPosition returns the position of the robot in that instant. The functionality subscribe returns the position of the robot peridically, requires a handleLocalizationChange that will be notified by LocalizationProvider.
\begin{figure}[ht!]
 \centering
 \includegraphics[scale=.43]{./RA-3/localization_CIP.png}
 \caption{Localization}
 \label{fig:localization_cip}
\end{figure}

\subsubsection{Navigation} %CIP-IMAGE: Done* missing parameters and returns. Needs some tweaks.
Figure \ref{fig:navigation_cip} describes navigation service's contract, interface and protocol. The navigation service provides three functionalities driveTo, executePath and stop. The funcionality driveTo has the robot drive to a position. The functionality executePath executes a path going to specific positions on the map. The functionality stop stops whatever navigation action it is doing.
\begin{figure}[ht!]
 \centering
 \includegraphics[scale=.43]{./RA-3/navigation_CIP.png}
 \caption{Navigation}
 \label{fig:navigation_cip}
\end{figure}

\subsubsection{Path Planning} %CIP-IMAGE: Done* missing parameters and returns.
Figure \ref{fig:pathplanning_cip} describes navigation service's contract, interface and protocol. The navigation service provides two functionalities, definePath and calculateDistance. The definePath funcionality defines the best path to a product area considering the position of other robots using the dijkstra algorithm requires the implementation of executePath. The functionality calculateDistance calculates the Manhattan distance to a certain product area.
\begin{figure}[ht!]
 \centering
 \includegraphics[scale=.43]{./RA-3/pathPlanning_CIP.png}
 \caption{Path Planning}
 \label{fig:pathplanning_cip}
\end{figure}

\subsubsection{Object Manipulation} %CIP-IMAGE: Done* missing parameters and returns.
Figure \ref{fig:objectmanipulation_cip} describes object manipulation sevice's contract, interface and protocol. The object manipulation service provides two functionalities pickUp and release. The functionality pickUp positions the robot to pick up the product and requires the handlePickUpResult. The functionality release releases the product on the ground and requires the handleReleaseResult. 
\begin{figure}[ht!]
 \centering
 \includegraphics[scale=.43]{./RA-3/objectManipulation_CIP.png}
 \caption{Object Manipulation}
 \label{fig:objectmanipulation_cip}
\end{figure}

\subsubsection{Communication} %CIP-IMAGE: Nope TODO: Rethink communication capabilities connections
% \begin{figure}[ht!]
%  \centering
%  \includegraphics[scale=.43]{./RA-3/communication_CIP.png}
%  \caption{Communication}
%  \label{fig:communication_cip}
% \end{figure}

\subsubsection{Control} %CIP-IMAGE: Done* TODO: Rethink Control functionalities, missing parameters and returns.
\begin{figure}[ht!]
 \centering
 \includegraphics[scale=.43]{./RA-3/control_CIP.png}
 \caption{Control}
 \label{fig:control_cip}
\end{figure}

\subsubsection{Robotic Agent} %CIP-IMAGE: Done* missing parameters and returns.
\begin{figure}[ht!]
 \centering
 \includegraphics[scale=.43]{./RA-3/roboticAgent_CIP.png}
 \caption{Robotic Agent}
 \label{fig:roboticagent_cip}
\end{figure}

\subsubsection{Application} %CIP-IMAGE: Nope
% \begin{figure}[ht!]
%  \centering
%  \includegraphics[scale=.43]{./RA-3/application_CIP.png}
%  \caption{Application}
%  \label{fig:application_cip}
% \end{figure}

\subsection{Robotic Service Composition}
%TODO: Service Architecture

\section{Step RA-4: Robotic Architecture Detailing}
\subsection{Implementation Strategies}
This section shows the implementation strategies for the services modeled in the previous section. Some services might be totally or partially aquired from open online repositories. The implementation strategy might change according to a necessity or limitation occuring later during project implementation.

\subsubsection*{Camera Sensor Service}
%Determine camera - options: Actual Pioneer camera or Jackal's Camera
\subsubsection*{Sonar Sensor Service}
The sonar sensor service is an abstraction of the sixteen sonars. The sonars are positioned on the front and the rear of the robot, eight on the front and eight on the rear, each with a different angle. The service communicates directly with the sensors, gathers them, and organizes in structured data providing the distance registred by each sonar.
%Already implemented in the pioneer?
\subsubsection*{Fork Lift Actuator Service}
The fork lift actuator service controls the fork lift directly. The fork lift only moves in one axis, up and down.
%ros control prismatic joint - already implemented
\subsubsection*{Drive Actuator Service}
The drive actuator service controls the motors of the two wheels responsible for moving the robot. The service will implement differential drive and provide an abstraction for easily moving the robot, through only linear and angular movement.

\subsubsection*{Image Processing Service}
The image processing service processes the images provided by the camera service providing the identification of the guide lines. The service takes in raw images and provides structured data identifying the guide lines and line crossings. The processed data identifies the relational position of the lines according to the robot, both linearly and angularly.

\subsubsection*{Map Information Service}
The map information service stores the map shared between all the robots. The map stores the guide lines grid in a graph representation. The map contains the position of each robot and product area in the graph.

\subsubsection*{Localization Service}
The localization service looks into the camera images to provide the position of the robot in the map. The localization service takes in the processed images identifying the guide lines and crossings to determine, once it sees a crossing, what path it took to update it's actual position.

\subsubsection*{Navigation Service}
The navigation service is responsible for keeping the robot following the line and take the correct turns once it reaches a crossing. The service takes in the processed camera images identifying the relational position of the robot according to the guide lines and turns the robot right, left of keeps going straight to successfully follow the guide line. The service also executes paths generated by the path planning service. It takes the path and checks the actual localization and the next step necessary to keep following the path, go straight, turn right, go straight, or occasionally reverse direction.

\subsubsection*{Path Planning Service}
The path planning service calcultes the manhattan distance between the map positions and determines the shotest path between two map postisions. To determine the best path between two map positions the service uses the graph representation of the map to apply the dijkstra algorithm.

\subsubsection*{Object Manipulation Service}
The object manipulation abstracts the manipulation of the object, picking up and releasing. The object manipulation service makes uses the sonar sensor service data to localizate the position of the product in relationship to the robot, then it moves the fork lift to pick the product up, navigates to it and positions the fork in the correct position for lifting it up. The last step is to lift the product to carrying to position. 
\subsubsection*{Communication Service}
%probably going to be removed
\subsubsection*{Operation Service}
The operation service provides an interface for an operator to manually drive the robot. The service sends messages to the robot as the operator presses the arrows or awsd keys of the keyboard in the computer that has access to the terminal that where the service is active.

\subsubsection*{Control Service}
The control service coordinates the services navigation service and object manipulation service. The service communicates the product area where the product must be transported from to the navigation service, once the robot is done navigating to the product area the control uses the object manipulation service to pick the object up. Then it communicates the product area where the product must be delivered to the navigation service, again, once it is done, uses the object manipulation service to release the object.

\subsubsection*{Robotic Agent Service}
The robotic agent service is the abstraction of the robot itself. The service communicates with other robots to determine the cloosest one to the product and internalizing the transport in the robot.

\subsubsection*{Application Service}
The application service is the interface with the operator to request a new transport. The service communicate to all robots to decide which robot will do the transport and the product areas where the must be transported from and delivered. 

\subsection{Deployment}

\bibliographystyle{plain}
\bibliography{references}
\end{document}
